\documentclass[12pt]{article}
\usepackage{url,graphicx,tabularx,array,geometry,enumitem,amsmath}
\setlength{\parskip}{1ex} %--skip lines between paragraphs
\setlength{\parindent}{0pt} %--don't indent paragraphs

%-- Commands for header
\renewcommand{\title}[1]{\textbf{#1}\\}
\renewcommand{\line}{\begin{tabularx}{\textwidth}{X>{\raggedleft}X}\hline\\\end{tabularx}\\[-0.5cm]}
\newcommand{\leftright}[2]{\begin{tabularx}{\textwidth}{X>{\raggedleft}X}#1%
& #2\\\end{tabularx}\\[-0.5cm]}

%\linespread{2} %-- Uncomment for Double Space
\begin{document}

\title{Digital Signal Processing - Assignment 4}
\line
\leftright{\today}{Stephanie Lund (2555914)\\Aljoscha Dietrich(2557976)} %-- left and right positions in the header

\section*{Exercise 1}

\subsection*{1.1}
The LPC technique predicts the next values from a given signal. It is based on the source-filter model of speech production, which states that speech is created by a source (the vocal cords) and an independent filter (the vocal tract) which creates resonances (formants). It is useful for encoding compressed speech.

\subsection*{1.2}
TODO: Block diagram
\begin{align*}
	x_{j+1} = \sum_{i=0}^{k} a_k x_{j-i}
\end{align*}

\subsection*{1.3}
TODO

\subsection*{1.4}
TODO

\subsection*{1.5}
TODO

\subsection*{1.6}
TODO

\end{document}