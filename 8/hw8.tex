\documentclass[12pt]{article}
\usepackage{url,graphicx,tabularx,array,geometry,enumitem,amsmath}
\setlength{\parskip}{1ex} %--skip lines between paragraphs
\setlength{\parindent}{0pt} %--don't indent paragraphs

%-- Commands for header
\renewcommand{\title}[1]{\textbf{#1}\\}
\renewcommand{\line}{\begin{tabularx}{\textwidth}{X>{\raggedleft}X}\hline\\\end{tabularx}\\[-0.5cm]}
\newcommand{\leftright}[2]{\begin{tabularx}{\textwidth}{X>{\raggedleft}X}#1%
& #2\\\end{tabularx}\\[-0.5cm]}
\def\ci{\perp\!\!\!\perp}


%\linespread{2} %-- Uncomment for Double Space
\begin{document}

\title{Digital Signal Processing - Assignment 8}
\line
\leftright{\today}{Stephanie Lund (2555914)\\Stalin Varanasi (2556235)} %-- left and right positions in the header

\section*{Exercise 1}

\subsection*{1.1}
We can use the error function:
\begin{align*}
e(n) = s(n) - \hat{s}(n) = s(n) - h(n) \ast x(n)
\end{align*}

converted to the frequency domain to find $J(\omega)$. Using the properties of additivity, and that convolution in the time domain is multiplication in the frequency domain, this is:
\begin{align*}
J(\omega) = S(\omega) - H(\omega)X(\omega)
\end{align*}

\subsection*{1.2}
\begin{align*}
E[J(\omega)^2] &= E[(S(\omega) - H(\omega)X(\omega))^2] \\
&= (S(\omega))^2 - 2S(\omega)H(\omega)X(\omega) + (H(\omega)X(\omega))^2 \\
\end{align*}

TODO: take the derivative $\frac{\partial J}{\partial H(\omega)}$, set to zero, somehow algebra it into the given equation\\

\subsection*{1.3}
\begin{align*}
H(\omega) = \frac{\Phi_{sx}(\omega_k)}{\Phi_{ss}(\omega_k) \Phi_{nn}(\omega_k)}
\end{align*}

TODO: figure out how to handle the numerator

\subsection*{1.4}
should just be multiply by $X(\omega)$ and take the inverse fourier transform? double-check this

\subsection*{1.5}
TODO, answer 1.3 is probably helpful

\subsection*{1.6}
TODO

\end{document}